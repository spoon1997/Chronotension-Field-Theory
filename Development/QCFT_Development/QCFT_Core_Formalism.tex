\documentclass[12pt]{article}
\usepackage{amsmath, amssymb, hyperref}
\usepackage[a4paper, margin=1in]{geometry}
\usepackage{physics}
\usepackage{bm}

\title{Quantum Chronotension Field Theory (QCFT): Core Formalism}
\author{Luke Cann and GPT Collaboration}
\date{\today}

\begin{document}
\maketitle

\section*{1. Field Content}

Let $\eta^a(x, t)$ be a real-valued, vector quantum field in internal index space $a = 1, \dots, N$.

It replaces the metric tensor of General Relativity and the gauge fields of the Standard Model. All physical structure arises from its evolution and topology.

\section*{2. QCFT Lagrangian}

\begin{equation}
\mathcal{L}_{\text{QCFT}} = \frac{1}{2} \delta^{ab} \partial_\mu \eta^a \partial^\mu \eta^b - \lambda (\eta^a \eta^a - v^2)^2 + \theta \epsilon^{\mu\nu\rho\sigma} f_{\mu\nu}^a f_{\rho\sigma}^a
\end{equation}

Where:
\begin{itemize}
  \item $\eta^a$ is the time-viscosity vector field
  \item $f_{\mu\nu}^a = \partial_\mu \eta^a \partial_\nu \eta^a - \partial_\nu \eta^a \partial_\mu \eta^a$
  \item $\lambda$ is the quartic self-interaction
  \item $\theta$ governs topological (non-perturbative) dynamics
\end{itemize}

\section*{3. Solitons and Chronodes}

Localized topological excitations in $\eta^a$ are stable and quantized. These are called \textbf{chronodes}, and they correspond to the particle spectrum:

\begin{itemize}
  \item Fermions emerge from knot topology, twist, and braiding in $\eta^a$
  \item Bosons emerge as propagating curvature or η-exchange structures
  \item Generations = harmonic modes of underlying chronode types
\end{itemize}

\section*{4. Quantization}

Canonical quantization is imposed via:

\begin{equation}
[\hat{\eta}^a(x), \hat{\pi}_\eta^b(y)] = i \hbar \delta^{ab} \delta^3(x - y)
\end{equation}

Fock space can be constructed from soliton states and wave packets of $\eta^a$.

\section*{5. Interactions and Feynman Rules}

Propagator:

\begin{equation}
\Delta^{ab}(k) = \frac{i \delta^{ab}}{k^2 + i\epsilon}
\end{equation}

Vertices:
\begin{itemize}
  \item 3-point: $i g_{abc}$
  \item 4-point: $-4! i \lambda$
  \item Topological $\theta$-term: suppressed in perturbation theory
\end{itemize}

\section*{6. S-Matrix Formalism}

Scattering amplitudes derived from correlators of $\delta \eta^a$ using LSZ-like reduction:

\begin{equation}
S_{fi} = \int \prod_j d^4x_j \, e^{i p_j \cdot x_j} (\Box_{x_j}) \langle 0 | T \{ \delta \eta^{a_1}(x_1) \dots \delta \eta^{a_n}(x_n) \} | 0 \rangle
\end{equation}

\section*{7. Renormalization}

Standard counterterm structure:
\[
Z_\eta (\partial \eta)^2 - Z_\lambda \lambda (\eta^a \eta^a - v^2)^2
\]
Renormalization is possible at all orders. No divergences in topological solitons.

\section*{8. Gauge Emergence}

Internal topology and field curvature give rise to:
\begin{itemize}
  \item SU(3) color from braiding and winding
  \item SU(2) × U(1) from field cross-couplings
  \item Higgs mechanism: radial fluctuation of $|\eta^a|$
\end{itemize}

\section*{9. Summary}

QCFT offers:
\begin{itemize}
  \item A unified theory of particles, interactions, and spacetime
  \item Solitonic basis for Standard Model fields
  \item Quantum consistency, unitarity, and renormalizability
  \item Reinterpretation of cosmology, gravity, and black holes via $\eta(x,t)$
\end{itemize}

\end{document}