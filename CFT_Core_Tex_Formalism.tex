
\documentclass{article}
\usepackage{amsmath, amssymb, geometry, graphicx, hyperref}
\geometry{margin=1in}
\title{Chronotension Field Theory (CFT): Core Formalism and Mathematical Framework}
\author{Luke W. Cann}
\date{2025}

\begin{document}

\maketitle

\section*{Abstract}
Chronotension Field Theory (CFT) replaces spacetime geometry with a scalar temporal resistance field \(\eta(x,t)\). This document contains the foundational mathematics of CFT: field equations, Lagrangians, quantization rules, and effective curvature analogues. It is designed as a reference core for peer testing and long-term archival.

\section{CFT Lagrangian and Field Equations}

\subsection{Lagrangian Density}
\begin{equation}
\mathcal{L}_\eta = \frac{1}{2} \eta^2 ( -\dot{\eta}^2 + |\nabla \eta|^2 ) - V(\eta)
\end{equation}

\subsection{Potential Term}
\begin{equation}
V(\eta) = \lambda (\eta^2 - 1)^2
\end{equation}

\subsection{Euler-Lagrange Field Equation}
\begin{equation}
\eta^2 (\ddot{\eta} - \nabla^2 \eta) + 3\eta ( -\dot{\eta}^2 + |\nabla \eta|^2 ) + \frac{dV}{d\eta} = 0
\end{equation}

\section{Stress-Energy and Hamiltonian Density}

\subsection{Stress-Energy Tensor}
\begin{equation}
T^{\mu \nu} = \eta^2 \partial^\mu \eta \partial^\nu \eta - g^{\mu \nu} \mathcal{L}_\eta
\end{equation}

\subsection{Hamiltonian Density}
\begin{equation}
\mathcal{H}_\eta = \frac{1}{2} \eta^2 |\nabla \eta|^2 - \frac{3}{2} \eta^2 \dot{\eta}^2 + V(\eta)
\end{equation}

\section{Quantum Chronotension Field Theory (QCFT)}

\subsection{Quantization Rules}
\begin{align}
\hat{\eta}(x,t), \quad \hat{\pi}_\eta(x,t) &= -\eta^2 \dot{\eta} \\
[\hat{\eta}(x), \hat{\pi}_\eta(y)] &= i\hbar \delta(x-y)
\end{align}

\section{Chronode Equation (Solitonic Solutions)}
\begin{equation}
\eta^2 \eta'' + 3\eta (\eta')^2 = \frac{dV}{d\eta}
\end{equation}

\section{Effective Curvature and Geodesics}

\subsection{Effective Ricci-like Curvature}
\begin{equation}
R_{\text{eff}} \sim -\nabla^2 \eta
\end{equation}

\subsection{Geodesic Deviation via \(\eta\)}
\begin{equation}
\frac{d^2 x^\mu}{d\tau^2} \sim \partial^\mu \eta
\end{equation}

\subsection{Effective Metric (Suggested)}
\begin{equation}
ds^2 = -\frac{1}{\eta^2} dt^2 + dx^2 + dy^2 + dz^2
\end{equation}

\section{Cosmological Implications}

\subsection{Observed Redshift Remapping}
\begin{equation}
z_{\text{actual}} \approx \eta(z) - 1
\end{equation}

\subsection{Distance Modulus Correction}
\begin{equation}
\mu_{\text{CFT}} = \mu_{\text{obs}} + 5 \log_{10} (\eta(z))
\end{equation}

\subsection{BAO Rescaling}
\begin{equation}
d_{\text{CFT}} = d_{\text{GR}} / \eta(z)
\end{equation}

\section{Final Notes}
This document formalizes the core equations of Chronotension Field Theory for publication, peer validation, and simulation reference. Experimental validation focuses on SN1a light curve stretch, clock drift in low-\(\eta\) zones, BAO compression, and CMB amplitude restoration.

\end{document}
