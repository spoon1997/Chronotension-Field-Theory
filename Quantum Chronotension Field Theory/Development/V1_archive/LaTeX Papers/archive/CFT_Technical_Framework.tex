
\documentclass[12pt]{article}
\usepackage{amsmath, amssymb}
\usepackage{geometry}
\geometry{margin=1in}
\title{Chronotension Field Theory (CFT)\\Technical Framework \& Clarifications}
\author{Luke W. Cann}
\date{July 2025}

\begin{document}
\maketitle

\section*{I. Core Field Structure}

CFT models the universe as a continuous, time-fluid substrate characterized by two scalar fields:

\begin{itemize}
  \item \textbf{Viscosity Field} \( \eta(x, t) \): Governs resistance to time flow.
  \item \textbf{Tension Field} \( \mathcal{T}(x, t) \): Governs structural gradients and directional pressure in the field.
\end{itemize}

These fields generate observable gravitational and cosmological phenomena through their spatial and temporal gradients.

\subsection*{1.1 Viscosity Decay}
\[
\eta(t) = \eta_0 \cdot \exp\left[ -\left( \frac{t}{t_c} \right)^\beta \right]
\]

\subsection*{1.2 Tension Field Form}
\[
\mathcal{T}(x, t) = T_0 \cdot \exp\left[ -\left( \frac{r}{\alpha} \right)^\beta \right]
\]

\subsection*{1.3 Observer Time Remapping}
\[
t_{\text{obs}} = \int f(\eta(t)) \, dt \quad \text{with} \quad f(\eta) = \frac{0.6}{1 + \eta} + \frac{0.4}{1 + e^{2(\eta - 0.5)}}
\]

\section*{II. Expansion Dynamics}

\subsection*{2.1 Scale Factor Model}
\[
a(t) = a_0 \cdot \exp\left[ f(t; \eta(t)) \right]
\]

\subsection*{2.2 BAO Ruler Derivation}
\[
r_s^{\text{CFT}} = \int_0^{t_{\text{drag}}} \frac{c_s}{\sqrt{1 + \eta(t)}} \, dt \approx 94.4 \, \text{Mpc}
\]

\section*{III. Quantum Chronotension (C-QFT)}

\subsection*{3.1 Lagrangian}
\[
\mathcal{L}_{\text{CFT}} = -\frac{1}{2} \mathcal{T}(x,t) \, \partial^\mu \eta \, \partial_\mu \eta - V(\eta) + \mathcal{L}_{\text{int}}(\eta, \chi)
\]

\subsection*{3.2 Modified Uncertainty Principle}
\[
\Delta x \cdot \Delta(\partial_x \eta) \geq \hbar_\eta
\]

\subsection*{3.3 Chronode Dynamics}
Chronodes are soliton-like solutions of the \( \eta \) field. Interactions are governed by tension energy and conserved \( \eta \)-flux.

\section*{IV. Effective Geometry \& Observational Predictions}

\subsection*{4.1 Effective Metric}
\[
ds^2 = \frac{1}{\eta(x,t)} dt^2 - a(t)^2 dx^2
\]

\subsection*{4.2 Stress-Energy Tensor}
\[
T^{\mu\nu}_{\text{CFT}} = \partial^\mu \eta \, \partial^\nu \eta - g^{\mu\nu} \left( \frac{1}{2} \mathcal{T} (\partial \eta)^2 + V(\eta) \right)
\]

\subsection*{4.3 Gravitational Lensing}
Deflection arises from \( \eta(x,t) \) curvature acting like a refractive index gradient, bypassing conventional spacetime curvature.

\section*{V. Falsifiability and Predictions}

\subsection*{5.1 Falsification Criteria}
\begin{itemize}
  \item H(z) mismatch after remapping
  \item BAO peak unmatchable without dark matter
  \item Failure to match SN Ia residuals post-remap
\end{itemize}

\subsection*{5.2 Unique Predictions}
\begin{itemize}
  \item Mid-\( \ell \) CMB suppression (\( \ell \approx 30-200 \))
  \item Observable redshift anomalies via \( \eta \)-field flow
  \item Non-GR lensing profiles in high-viscosity zones
\end{itemize}

\end{document}
