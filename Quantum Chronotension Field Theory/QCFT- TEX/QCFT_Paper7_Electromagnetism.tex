
\documentclass[12pt]{article}
\usepackage{amsmath, amssymb}
\usepackage{geometry}
\usepackage{graphicx}
\usepackage{physics}
\usepackage{hyperref}
\geometry{margin=1in}
\title{Quantum Chronotension Field Theory – Paper VII\\Emergence of Electromagnetism from the $\eta$-Field}
\author{Luke W. Cann, Independent Theoretical Physicist and Founder of QCFT}
\date{}

\begin{document}
\maketitle

\begin{abstract}
This paper reinterprets classical and quantum electromagnetism through the lens of Quantum Chronotension Field Theory (QCFT). Electromagnetic phenomena arise not from fundamental gauge fields in spacetime, but as emergent behaviors of structured oscillations, rotations, and topologies within the time-viscosity field $\eta^a(x,t)$. Charge, light, and field propagation are unified under QCFT's solitonic chronode framework. Maxwell's equations emerge as low-energy limits of $\eta^a$ dynamics.
\end{abstract}

\section{Foundations}

\subsection{The $\eta$-Field and Its Vectorization}

QCFT generalizes the scalar time-viscosity field $\eta(x,t)$ to a vector-valued field $\eta^a(x,t)$. The Lagrangian is:
\[
\mathcal{L}_{\text{QCFT}} = \frac{1}{2} \delta^{ab} \partial_\mu \eta^a \partial^\mu \eta^b - \lambda (\eta^a \eta^a - v^2)^2 + \theta \epsilon^{\mu\nu\rho\sigma} f^a_{\mu\nu} f^a_{\rho\sigma}
\]
with
\[
f^a_{\mu\nu} = \partial_\mu \eta^a \partial_\nu \eta^a - \partial_\nu \eta^a \partial_\mu \eta^a
\]

\subsection{Emergent Spacetime}

Apparent metric structure arises from $\eta$:
\[
ds^2 = -\frac{dt^2}{\eta^2(x,t)} + \eta^2(x,t) dx^i dx^i
\]
Wave speed becomes $c(x,t) = 1/\eta(x,t)$.

\section{Light as Compression Waves}

Light is a longitudinal compression wave in the $\eta$-field:
\[
c(x,t) = \frac{1}{\eta(x,t)}
\]
Polarization corresponds to planar or helical oscillations in $\eta^a$ components.

\section{Charge and Fields}

\subsection{Chronodes as Topological Sources}

Charged particles are chronodes — stable $\eta^a$ knots. Electric charge is winding in the $\eta^a$ field:
\begin{itemize}
\item Electron: U(1) winding, negative
\item Proton: Composite chronode with counter-windings
\end{itemize}

\subsection{Electric and Magnetic Fields}

\begin{itemize}
\item $\vec{E}$: spatial gradient of $\eta^a$
\item $\vec{B}$: curl of $\eta^a$ rotational flow
\end{itemize}

\section{Maxwell's Equations}

In the low-energy limit:
\[
\nabla \cdot \vec{E} = \rho \quad ; \quad \nabla \times \vec{E} = -\frac{\partial \vec{B}}{\partial t}
\]
\[
\nabla \times \vec{B} = \mu_0 \vec{J} + \mu_0 \epsilon_0 \frac{\partial \vec{E}}{\partial t}
\]

These emerge from topological continuity and rotational inertia in $\eta^a$.

\section{Implications}

\begin{itemize}
\item Light speed varies in high-Gradia zones
\item Birefringence near black holes
\item Phase shifts from $\eta$-anisotropy
\item Nonlinear QED deviations at high tension
\end{itemize}

\section*{Conclusion}

Electromagnetism emerges from topological features of $\eta^a$. Light is a compression wave. Fields are gradients and circulation of the time field. Maxwell's equations are coarse-grained projections of field geometry.

\begin{center}
\textit{Electromagnetism is not a force. It is the ripple of braided time.}
\end{center}

\end{document}
