
\documentclass[12pt]{article}
\usepackage{amsmath, amssymb}
\usepackage{geometry}
\usepackage{booktabs}
\usepackage{hyperref}
\geometry{margin=1in}
\title{Quantum Chronotension Field Theory – Paper XI\\QCFT vs General Relativity}
\author{Luke W. Cann, Independent Theoretical Physicist and Founder of QCFT}
\date{}

\begin{document}
\maketitle

\begin{abstract}
This paper compares Quantum Chronotension Field Theory (QCFT) with General Relativity (GR), outlining their contrasting ontologies, interpretations of cosmological data, and gravitational predictions. GR treats gravity as curvature in spacetime geometry. QCFT replaces geometry with a dynamic time-viscosity field eta(x,t), where gradients (Gradia) generate gravitational effects. This shift eliminates the need for spacetime expansion, singularities, or dark matter halos.
\end{abstract}

\section{Ontological Foundations}

\begin{tabular}{@{}llll@{}}
\toprule
Aspect & General Relativity & QCFT \\
\midrule
Substrate & 4D spacetime manifold & Scalar eta(x,t) field \\
Geometry & Riemann curvature tensor & Emergent from eta-structure \\
Gravity & Geodesic deviation & Gradia (spatial gradient of eta) \\
Time & Coordinate dimension & Field with local viscosity \\
\bottomrule
\end{tabular}

\section{Redshift Interpretation}

GR defines redshift as:

\[
1 + z = \frac{a(t_{\text{obs}})}{a(t_{\text{emit}})} \quad \text{or} \quad 1 + z = \frac{1}{\sqrt{1 - 2GM/r}}
\]

QCFT defines redshift via eta decay:

\[
1 + z = \exp\left( \int_{\text{path}} \frac{d\eta(x,t)}{\eta(x,t)} \right)
\]

No scale factor or curvature required—redshift is a path integral over field decay and tension.

\section{Lensing and Structure Formation}

\begin{itemize}
\item GR: Curved spacetime bends light.
\item QCFT: Wavefronts refract along Gradia gradients.
\item Dense Gradia = filamentation; destructive zones = voids.
\end{itemize}

\section{Black Hole Comparison}

\begin{tabular}{@{}llll@{}}
\toprule
Property & GR Black Hole & QCFT Collapse (FCE) \\
\midrule
Nature & Singularity & Field rupture \\
Mass Location & Pointlike & Distributed tension \\
Escape & Forbidden & Eta-waves may exit \\
Interior & Undefined & Dynamic eta topology \\
\bottomrule
\end{tabular}

\section{Horizon and Causality}

\begin{itemize}
\item GR: Causal horizon from metric expansion.
\item QCFT: Horizon = field coherence boundary.
\item No inflation required—CMB isotropy from early eta connectivity.
\end{itemize}

\section{Expansion vs. Decay}

\begin{tabular}{@{}llll@{}}
\toprule
Phenomenon & GR View & QCFT View \\
\midrule
SN1a Redshift & Metric acceleration & Eta decay + Gradia tension \\
BAO Compression & Length scale stretch & Eta path shortening \\
CMB Peaks & Acoustic echoes & Eta interference \\
Clock Drift & Time dilation & Gradia flow \\
\bottomrule
\end{tabular}

\section{Predictive Differences}

\begin{tabular}{@{}llll@{}}
\toprule
Domain & GR Prediction & QCFT Prediction \\
\midrule
Lensing & Mass-only curvature & Gradia-based lensing (mass-optional) \\
Voids & Matter underdensity & Scar zones from interference \\
Redshift Residuals & Noise or gravity & Eta-field topology \\
Black Hole Death & Hawking evaporation & FCEs \\
\bottomrule
\end{tabular}

\section*{Conclusion}

QCFT reinterprets GR’s successes by replacing spacetime curvature with field tension. All gravitational, cosmological, and redshift behavior is encoded in the gradients and decay of the eta-field. QCFT eliminates the need for inflation, dark energy, and singularities—offering a coherent, field-based alternative.

\begin{center}
\textit{QCFT does not curve space. It stretches time.}
\end{center}

\end{document}
