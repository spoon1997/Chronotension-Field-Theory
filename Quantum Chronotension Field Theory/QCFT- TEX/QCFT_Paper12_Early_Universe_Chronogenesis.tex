
\documentclass[12pt]{article}
\usepackage{amsmath, amssymb}
\usepackage{geometry}
\usepackage{booktabs}
\usepackage{hyperref}
\geometry{margin=1in}
\title{Quantum Chronotension Field Theory – Paper XII\\Early Universe and Chronogenesis}
\author{Luke W. Cann, Independent Theoretical Physicist and Founder of QCFT}
\date{}

\begin{document}
\maketitle

\begin{abstract}
Quantum Chronotension Field Theory (QCFT) replaces singularity-driven cosmology with a field-based model of early time structure. The origin of the universe is modeled as the emergence and decay of a coherent time-viscosity field, eta(x,t), where chronodes arise as stable topological solutions in a once-uniform temporal substrate. Redshift, structure, and causality are redefined through field dynamics, not metric expansion.
\end{abstract}

\section{The Eta-Origin Hypothesis}

QCFT begins with a spatially smooth, temporally dense field: $ \eta(x,t) \approx \eta_0 $. This is a high-viscosity equilibrium. No geometry, particles, or forces exist—only an undifferentiated temporal fluid.

As time flows, fluctuations destabilize the uniform eta field, initiating wavefronts, interference, and collapse. This sets the stage for chronode formation.

\section{Chronogenesis}

Chronodes form when eta-wave interference nodes stabilize with sufficient internal curvature and field tension. These solitonic knots represent the earliest persistent structures, replacing particle formation models and baryogenesis.

Chronogenesis is a phase transition in eta: from fluid field to structured soliton lattice.

\section{The Great Unfurl}

QCFT replaces the "Big Bang" with a smooth temporal decay:

\begin{center}
\textbf{The Great Unfurl:} $ \eta(t) \downarrow $, field tension increases, structure emerges.
\end{center}

All redshift observed today originates from this cumulative decay:

\[
1 + z = \exp\left( \int_{\text{photon path}} \frac{d\eta(x,t)}{\eta(x,t)} \right)
\]

Redshift reflects time-field transformation, not metric expansion.

\section{Resolving the Horizon Problem}

In QCFT, causal disconnection is a viscosity issue, not spatial separation. The maximum causal region is defined by the eta-horizon:

\[
\text{eta-horizon} \equiv \text{Region where } \eta(x,t) > \eta_{\text{crit}}
\]

Early uniformity of eta ensures all observable regions were once connected. No inflation is required.

\section{Eta-Wave Epoch and Structure Formation}

Early field fluctuations produce eta-waves that interfere and collapse, triggering node formation. These nodes concentrate curvature, forming stable chronodes and Gradia tension networks. The cosmic web is seeded by field interference, not gravitational collapse.

Voids form in regions of destructive interference—low-eta zones where chronodes cannot stabilize.

\section{Time's Asymmetry and Field Decay}

Time’s arrow arises from monotonic decay of eta:

\[
\frac{\partial \eta}{\partial t} < 0 \quad \Rightarrow \quad \text{Time tension increases}
\]

As eta decays, structure consolidates and irreversibility emerges. No entropy principle is required.

\section*{Conclusion}

QCFT describes the early universe not as a singularity, but as a smooth unfolding of temporal viscosity. Structure forms through chronodes in eta, not particles. Redshift and causality arise from field evolution. The inflationary model is replaced by a falsifiable, coherent time-dynamic alternative.

\begin{center}
\textit{The beginning was not a bang.\\It was the folding of silence into time.}
\end{center}

\end{document}
