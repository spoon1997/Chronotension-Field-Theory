
\documentclass[12pt]{article}
\usepackage{amsmath, amssymb}
\usepackage{geometry}
\usepackage{setspace}
\usepackage{physics}
\usepackage{graphicx}
\usepackage{hyperref}
\geometry{margin=1in}
\title{Quantum Chronotension Field Theory – Paper V\\\large Chronode Quantization and Interaction}
\author{Luke W. Cann, Independent Theoretical Physicist and Founder of QCFT}
\date{}

\begin{document}
\maketitle

\begin{abstract}
This paper formalizes the quantization of chronodes in Quantum Chronotension Field Theory (QCFT). Chronodes are solitonic excitations in the time-viscosity field $\eta^a(x,t)$ that encode mass, charge, and spin via topological structure. Their quantization leads to a complete, renormalizable quantum field theory without metric geometry. This paper develops the operator formalism, Fock space, interaction principles, and path integrals governing chronode dynamics and scattering.
\end{abstract}

\section{Introduction}

QCFT replaces background spacetime with a dynamic field $\eta^a(x,t)$. Chronodes are its stable topological excitations. Their quantization is essential for modeling quantum interactions, particle stability, and soliton dynamics.

\section{Topological Soliton Quantization}

Chronodes are stable, localized solutions to the QCFT field equations:

\[
\mathcal{L}_{QCFT} = \frac{1}{2} \delta^{ab} \partial_\mu \eta^a \partial^\mu \eta^b - \lambda (\eta^a \eta^a - v^2)^2 + \theta \epsilon^{\mu\nu\rho\sigma} f_{\mu\nu}^a f_{\rho\sigma}^a
\]

Canonical quantization proceeds via:

\[
[ \hat{\eta}^a(x), \hat{\pi}_b(y) ] = i \hbar \delta^a_b \delta(x - y)
\]

\section{Chronode States and Fock Space}

Chronode excitations are decomposed into normal modes:

\[
\hat{\eta}^a(x,t) = \sum_k \left( a_k^a u_k(x,t) + a_k^{a\,\dagger} u_k^*(x,t) \right)
\]

Operators $a_k^a$ and $a_k^{a\,\dagger}$ annihilate and create chronodes of mode $k$.

\section{Interaction Framework}

Interactions arise through soliton merging, splitting, and braiding. Topological quantities such as winding number and $\eta^2$ density are conserved. Virtual particles are replaced by real field transformations.

\section{Path Integral Formulation}

The partition function is:

\[
Z = \int \mathcal{D}\eta^a \, e^{i \int \mathcal{L}_{QCFT} \, d^4x}
\]

Topological sectors contribute independently, bypassing metric-based propagators.

\section{Scattering and the S-Matrix}

The S-matrix is formulated via topological transitions:

\[
S_{fi} = \langle \text{final} | \hat{U} | \text{initial} \rangle
\]

where $\hat{U}$ evolves $\eta^a$ between asymptotic states. No need for virtual particles or perturbative corrections.

\section{Comparison with Standard QFT}

\begin{itemize}
    \item \textbf{Preserved:} Locality, unitarity, causality, quantization
    \item \textbf{Rejected:} Background spacetime, virtual particles
    \item \textbf{Replaced:} Gauge symmetry emerges from topology
\end{itemize}

\section{Emergent Interaction Strengths}

Coupling strengths emerge from $\eta$-curvature:

\[
g_{\text{eff}} \sim \int \eta^a \nabla \eta^b \, d^3x
\]

\section{Chronode Stability and Resonance}

Stability arises from:
\begin{itemize}
    \item Core $\eta^2$ concentration
    \item Gradia field tension
    \item Minimal $\eta$-wave dissipation
\end{itemize}

Resonances like neutrino oscillation result from slow twisting of topological mode classes.

\section*{Conclusion}

Chronode quantization completes QCFT’s transition from classical field to quantum theory. Interactions, decay, and scattering are described as topological transformations, yielding a complete field-based model of particle physics and cosmology.

\begin{center}
\emph{Time is not discrete.\\
Time is braided.}
\end{center}

\end{document}
