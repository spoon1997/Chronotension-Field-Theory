
\documentclass{article}
\usepackage{amsmath}
\usepackage{geometry}
\geometry{margin=1in}

\title{Chronotension Field Theory --- Soliton Derivation Formulas}
\author{Luke W. Cann}
\date{}

\begin{document}

\maketitle

\section*{Purpose}
Formal derivation of soliton behavior in Chronotension Field Theory (CFT) via the sine-Gordon equation.

\section{CFT Lagrangian (Scalar Field Form)}

\[
\mathcal{L}_{\text{CFT}} = -\frac{1}{2} \mathcal{T}(x, t) \, \partial^\mu \eta \, \partial_\mu \eta - V(\eta)
\]

Where:
\begin{itemize}
  \item $\mathcal{T}(x,t)$: local tension scalar field
  \item $\eta(x,t)$: viscosity field (acts as scalar field)
  \item $V(\eta)$: potential energy of the time field
\end{itemize}

\section{Euler--Lagrange Equation}

\[
\partial_\mu \left( \frac{\partial \mathcal{L}}{\partial (\partial_\mu \eta)} \right) - \frac{\partial \mathcal{L}}{\partial \eta} = 0
\]

Assuming constant tension $\mathcal{T}(x,t) = T_0$, we simplify:

\[
T_0 (\partial_t^2 \eta - \partial_x^2 \eta) + \frac{dV}{d\eta} = 0
\]

\section{Sine-Gordon Potential}

To match the sine-Gordon equation:

\[
\partial_t^2 \eta - \partial_x^2 \eta + \sin(\eta) = 0
\]

We require:

\[
\frac{dV}{d\eta} = -T_0 \cdot \sin(\eta) \Rightarrow V(\eta) = T_0 \cdot \cos(\eta)
\]

Thus, the Lagrangian becomes:

\[
\mathcal{L}_{\text{SG-CFT}} = -\frac{1}{2} T_0 \, (\partial^\mu \eta \, \partial_\mu \eta) - T_0 \cdot \cos(\eta)
\]

\section{Soliton Solution (Kink)}

A single-kink soliton solution is:

\[
\eta(x,t) = 4 \arctan \left( \exp\left( \pm \gamma(x - vt - x_0) \right) \right)
\]

Where:
\begin{itemize}
  \item $v$: velocity of the soliton
  \item $\gamma = 1/\sqrt{1 - v^2}$: Lorentz factor
  \item $x_0$: center position of the kink
\end{itemize}

\section{Energy Density}

\[
\mathcal{E}(x) = \frac{1}{2} T_0 \left( (\partial_t \eta)^2 + (\partial_x \eta)^2 \right) + T_0 \cdot (1 - \cos(\eta))
\]

\section{Topological Charge (Q)}

For soliton stability:

\[
Q = \frac{1}{2\pi} \int_{-\infty}^{\infty} \partial_x \eta(x) \, dx = \pm 1
\]

\section*{Conclusion}

This confirms that chronodes can be fully described by soliton solutions when CFT is endowed with a sine-Gordon-type potential. Chronodes are therefore topologically stable structures in the viscosity field $\eta(x,t)$, supported by tension-mediated dynamics.

\end{document}
