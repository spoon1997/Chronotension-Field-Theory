
\documentclass{article}
\usepackage{amsmath}
\usepackage{amssymb}
\usepackage{geometry}
\geometry{margin=1in}

\title{Chronotension Field Theory: Metric Structure and Geometric Analogues}
\author{Luke W. Cann}
\date{2025}

\begin{document}

\maketitle

\section*{1. Overview}
Chronotension Field Theory (CFT) does not postulate spacetime curvature as the cause of gravity. Instead, it describes time as a dynamic, viscous field whose resistance to flow manifests as gravitational phenomena. However, to compare with General Relativity (GR), it is useful to formulate an equivalent metric structure.

\section*{2. Time-Flow Gradient and Viscosity}
We define the local time tension $\mathcal{T}(x,t)$ as a function of the viscosity field $\eta(x,t)$ and the gradient of time potential $\phi(x,t)$:

\[
\mathcal{T}(x,t) = \nabla \cdot \left( \frac{1}{\eta(x,t)} \nabla \phi(x,t) \right)
\]

This tension influences all physical observables by modulating the effective rate of time passage.

\section*{3. Effective Metric}
We define an effective metric $g_{\mu\nu}^{\text{CFT}}$ by encoding the time-viscosity field into the lapse function:

\[
ds^2 = -\left( \frac{1}{\eta(x,t)} \right)^2 dt^2 + a(t)^2 d\vec{x}^2
\]

This is a conformal rescaling of the standard Friedmann–Lemaître–Robertson–Walker (FLRW) metric, where $\eta(x,t)$ modifies the flow rate of proper time.

\section*{4. Geodesics in CFT}
A test particle follows a path of minimal tension resistance rather than shortest spacetime interval:

\[
\delta \int \sqrt{-\mathcal{T}(x,t)} \, ds = 0
\]

This leads to a modified geodesic equation influenced by the tension gradient $\nabla \mathcal{T}$ and viscosity $\eta(x,t)$.

\section*{5. Field Collapse Boundary}
As $\eta(x,t) \to 0$, the field becomes superfluid, and geodesics degenerate. This corresponds to the Field Collapse Event (FCE), where the metric effectively flattens:

\[
ds^2 \to -\infty \cdot dt^2 + a(t)^2 d\vec{x}^2
\]

Signaling instantaneous causal contact across all space — a hallmark of cyclic restart.

\section*{6. Interpretation}
While GR encodes curvature in $R_{\mu\nu}$, CFT encodes time-resistance in $\eta(x,t)$ and its spatial-temporal derivatives. The metric equivalent presented here allows comparisons to Einstein’s field equations, but CFT remains distinct in both ontology and dynamics.

\end{document}
